\documentclass[11pt,a4paper]{report}
\usepackage[margin=0.5in]{geometry}
\usepackage[explicit]{titlesec}
\usepackage[dvipsnames]{color}

\definecolor{mygray}{gray}{.75}

\titleformat{name=\section,numberless}[display]
  {\normalfont\scshape\Large}
  {\hspace*{-10pt}#1}
  {-15pt}
  {\hspace*{-110pt}\rule{\dimexpr\textwidth+80pt\relax}{2pt}\Huge}
\titlespacing*{\section}{0pt}{30pt}{10pt}

\titleformat{name=\subsection,numberless}[display]
  {\normalfont\scshape}
  {\hspace*{-10pt}#1}
  {-15pt}
  {\hspace*{-110pt}\rule{\dimexpr\textwidth+30pt\relax}{0.4pt}\Huge}
\titlespacing*{\subsection}{0pt}{20pt}{5pt}


\begin{document}

\noindent\Large\textbf{CM2101 (Human-Computer Interaction)}\\
\noindent\large\textit{Non-assessed Tutorial 1}
\vskip30pt

\noindent This tutorial covers some of the material taught over the first half of the module. 

\section*{Past exam questions}
These questions are taken from recent CM2101 exam papers. Each paper was made up of 60 marks, and the marks associated with each question are displayed for context. There are 64 marks' worth of questions in this tutorial. \textit{Note: some questions have been re-phrased.}

\begin{enumerate}

\item List and briefly describe five usability principles generally regarded as being important.\\
    \begin{flushright}(10 marks)\end{flushright}

\item Explain the notions of \textit{Gulfs of Execution and Evaluation} and their relevance in the context of interaction modelling of software systems.\\
    \begin{flushright}(8 marks)\end{flushright}

\item Describe the \textit{heuristic evaluation} technique and explain how it may be applied.\\
    \begin{flushright}(6 marks)\end{flushright}

\item Explain how the \textit{heuristic evaluation} method can address the \textit{Gulfs of Execution and Evaluation}.\\
    \begin{flushright}(4 marks)\end{flushright}

\item Identify the main elements of the user-centred design approach to user interface development.\\
    \begin{flushright}(6 marks)\end{flushright}

\item Compare and contrast the Waterfall approach and the user-centred iterative approach for the purposes of interaction design. Include the main characteristics of both approaches and discuss any advantages and limitations.
    \begin{flushright}(10 marks)\end{flushright}

\item Our human working memory is limited. Describe a relevant usability principle that addresses this limitation. In your answer, give examples of how the principle is implemented in the systems you've used.
    \begin{flushright}(10 marks)\end{flushright}

\item The Keystroke Level Model (KLM) is a quantitative usability evaluation method. Discuss the role of this method in informing the process of user interface design for interactive systems. Discuss its advantages and disadvantages.
    \begin{flushright}(10 marks)\end{flushright}

\end{enumerate}

\end{document}
